\documentclass[12pt,a4paper]{report}
\setlength\textwidth{145mm}
 \setlength\topmargin{0mm}
 \setlength\headsep{0mm}
 \setlength\headheight{0mm}
 
% Přepneme na českou sazbu
\usepackage[czech]{babel}
\usepackage[IL2]{fontenc}
\usepackage{graphicx} 

%% Použité kódování znaků: obvykle latin2, cp1250 nebo utf8:
\usepackage[utf8]{inputenc}


\newcommand{\ccc}[1]{{\mbox{\fontfamily{lmtt}\selectfont\textbf{#1}}}}

\begin{document}

\chapter*{Dokumentace}

Robot je reprezentovaný třídou \ccc{boe\_bot}, která udržuje kontext a poskytuje
přístup k senzorům a ovládání robota. K nalplánování trasy robota slouží rozhraní 
\ccc{planner}. Rozhraní je navrženo tak, aby v případě nutnosti poskytovalo možnost
různých implmentací plánovačů. Pro plánovač je nejprve nutné vstupní příkazy rozparsovat,
k tomuto účelu souží rozhraní \ccc{command\_parser}.

\section*{boe\_bot}
Tato třída reprezentuje stav, poskytuje zjednodušené ovládání robota a odstiňuje od
hardwaru. Před dlaším používáním je nutno třídu inicializovat zavoláním funkce \ccc{setup}.
Stavy jednotlivých senzorů lze získat skrze třídu \ccc{sensors}. Tlačítko
reprezentuje třída \ccc{push\_button}, která posktuje mimojiné funkce pro zjištění, zda je tlačítko
stlačeno či nikoliv. Třída \ccc{wheel\_control} reprezentuje pohon a poskytuje funkce pro
ovládání rychlosti kol a její vyhlazování. 

\section*{Plánovač}
Plánovač má za úkol nalézt správnou cestu z počáteční do cílové pozice. Cesta je
rozdělena na menší části, kde pro každou tuto část vrátí plánovač obsluhu robota 
reprezentovanou třídou \ccc{command}. Plánovač definuje následující rozhraní:
\begin{itemize}
\item  \ccc{prepare\_route} -- nastaví, aby plánovač nalezl vhodnou cestu z výchozí
do cílové pozice,
\item \ccc{get\_next\_command} -- vrátí příkaz pro robota řídící robota po část cesty.
\end{itemize}

Příkaz reprezentovaný třídou \ccc{command} obsahuje svůj vnitřní stav, dle
kterého probíhá řízení robota. Rozhraní příkazu tvoří následující funkce:
\begin{itemize} 
\item \ccc{update} -- v této funkci, která musí být volána v každé iteraci smyčky, je možné
pracovat s robotem,
\item \ccc{is\_done} -- funkce vrátí, zda je už příkaz dokončen.
\end{itemize} 

\subsection*{\ccc{square\_grid\_planner}}
Tato třída implementuje plánovač na čtvercové mřížce, tak jak je požadováno v zadání.
Plánovač vrací příkazy pro pohyb vpřed a otočení o 90 stupňů. Aby byl plánovač univerzální
pro různe roboty deklaruje abstraktní funkce pro získání příkazu pro posun a otočení. 

Pro reprezentaci pozice na mřížce slouží třída \ccc{position}, kde souřadnice jsou počítány
od nuly, tj. A1 je pozice $(0,0)$. Směr robota je určen výčtem \ccc{direction}. 
Pozici a otočení robota reprezentuje třída \ccc{location}, která se skládá z předchozích
dvou tříd. Implementaci příkazů pro boe bota poskytuje nádledující plánovač.

\subsection*{\ccc{boe\_bot\_planner}}
Tato třída dědí z předchozí a poskytuje pouze instance příkazů rodičovské implementaci.
Pro posun vpřed slouží třída \ccc{move} a pro otočení o $90$ stupňů slouží třída \ccc{turn}.
Těmto příkazům je předán ukazatel na instanci třídy \ccc{boe\_bot} zkrze kterého mohou 
ovládat robota.

\subsection*{\ccc{move}}
Tato třída reprezentuje příkaz, který řídí robota rovně kupředu po čáře od jednoho kříže po následující. 
Po dosažení následujícího kříže jede robot ještě mírně kupředu tak, aby byl po 
dokončení příkazu co možná nejpřesněji kolami na kříži. Při detekci kříže je také
prováděna detekce okraje čtvercového plátna a tuto informaci mouhou využívat další příkazy.

\subsection*{\ccc{turn}}
Tato třída reprezentuje příkaz, který otočí robotem o $90$ až $180$ stupnů dle aktuální pozice.
Otáčení je rozděleno do několika částí:
\begin{itemize}
\item sjetí z první čáry,
\item pohyb po bílé ploše,
\item najetí na druhou čáru.
\end{itemize}

Takto nastaveným chováním může nastat otočení o $180$ stupňů na okrajích čtvercového pole.
Plánovač počítá vždy s otočením o $90$ stupňů, tudíž jsou v takovém případě vygenerovány
dva příkazy otočení. K detekci nadbytečného otoční je použita informace z předchozích
pohybů popsána v předchozí sekci.

\section*{\ccc{command\_parser\_eeprom}}
Tato třída slouží k parsování vstupních textových příkazů uložených v paměti eeprom.
Také se stará o ukládání přikazů příchozích po sériové lince nebo napevno uložených 
v kódu. K samotnému parsování textu z paměti třída obsahuje konečný automat. 
S parserem se pak pracuje pomocí následujících funkcí:
\begin{itemize}
\item \ccc{fetch\_initial} -- načte počáteční pozici,
\item \ccc{fetch\_next} -- načte následující příkaz,
\item \ccc{get\_initial\_location} -- vrátí počáteční pozici a směr, 
\item \ccc{get\_current\_target} -- vrátí právě načtenou cílovou pozici příkazu, 
\item \ccc{is\_first\_directionX} -- vrátí, zda se má robot pohybovat nejprve po ose X pro právě načetný příkaz,
\item \ccc{get\_finish\_time\_constrain} -- vrátí časový požadavek právě načeného příkazu,
\item \ccc{reset\_commands} -- vrátí parser do počátečního stavu. 
\end{itemize}


\end{document}